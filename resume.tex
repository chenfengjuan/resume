% !TEX TS-program = xelatex
% !TEX encoding = UTF-8 Unicode
% !Mode:: "TeX:UTF-8"
\documentclass{resume}
\usepackage{zh_CN-Adobefonts_external} % Simplified Chinese Support using external fonts (./fonts/zh_CN-Adobe/)
%\usepackage{zh_CN-Adobefonts_internal} % Simplified Chinese Support using system fonts
\usepackage{linespacing_fix} % disable extra space before next section
\usepackage{cite}
\usepackage{geometry}   %设置页边距的宏包
%\usepackage{titlesec}   %设置页眉页脚的宏包

\usepackage{fancyhdr}
% \pagestyle{fancy}

\usepackage{graphicx}
\usepackage{multicol}
\usepackage{multirow}
\usepackage{tabu}
\usepackage{progressbar}
\geometry{top = 2.5cm}
\begin{document}
\pagenumbering{gobble} % suppress displaying page number

\Large{
  \begin{tabu}
    \scshape{陈凤娟} & \\
    & \email{cfj1601175228@163.com} & \pbar{Vue~}{0.9} \\
    & \phone{(+86) 15821579606} & \pbar{React~}{0.6} \\
    & \linkedin [segmentfault/陈凤娟]{https://segmentfault.com/u/chenfengjuan_59eadc7bee8f8} & \pbar{Javascript~}{0.9} \\
    & \github[github.com/chenfengjuan]{https://github.com/chenfengjuan} & \pbar{D3~}{0.7}
  \end{tabu}
}

\normalsize
\section{\faInfo\  求职意向}
\datedsubsection{\textbf{ 高级前端开发工程师}, 全职, 北京}{}
\textit{月薪25K-28K}\ 半个月到岗

\section{\faGraduationCap\  教育背景}
\datedsubsection{\textbf{华中科技大学}, 软件工程, 软件学院}{2011.9-2015.6}

\section{\faGraduationCap\  工作经历}

\datedsubsection{\textbf{           人人车}, 中级前端开发工程师}{2018.4 - 至  今}
\begin{onehalfspacing}
\begin{itemize}
  \item 负责公司大数据方向产品的前端,主要负责开发开发Jarvis(PC和APP)、实时计算、即席查询等项目;
  \item 以及大数据方向产品前端(包含日志采集、元数据管理、ETL等项目)的需求评审、任务划分、排期、进度跟踪、跨团队沟通、项目的各环境部署和上线等;
  \item 参与公司前端基础物料的开发与维护、文档编写;
  \item 带领应届生做项目;
\end{itemize}
\end{onehalfspacing}

\datedsubsection{\textbf{上海景栗科技有限公司}, 前端开发工程师}{2016.9-2018.3}
\begin{onehalfspacing}
\begin{itemize}
  \item 负责BI大数据系统的前端开发和交互设计;
  \item 学习大数据相关产品知识,和团队一起完善BI项目,为公司的各部门运营提供便捷的周报日报,统计分析工具;
  \item 参与公司微信社群的运营的SASS系统部分模块的前端开发;
\end{itemize}
\end{onehalfspacing}

\datedsubsection{\textbf{上海海隆软件有限公司}, 软件开发工程师}{2015.9-2016.9}
\begin{onehalfspacing}
负责日本花旗银行系统的维护和更新,以及子项目的前端开发。
\end{onehalfspacing}

\section{\faUsers\ 项目经历}

\datedsubsection{\textbf{Jarvis}}{人人车, 2018.4-至今}
\begin{onehalfspacing}
Jarvis实现了从\textbf{数据接入整合,到数据处理、分析、挖掘,再到多终端可视化},对数据进行全价值链管理的一站式\textbf{数据管理和分析}平台。
{ c l r }\textbf{技术栈:}Vue, VueRouter, Vuex, D3.js, Zrender.js, ElementUI, Sass, Eslint, StyleLint
\begin{itemize}
  \item 基于\textit{D3}、\textit{zrender}、和\textit{vue}实现的支持\textit{svg}和\textit{canvas}两种模式的\textbf{图表库};
  \item 图表编辑: 任意纬度的拖拽数据、排序、聚合、高级计算等\textbf{数据可视化}分析功能,公式编辑计算字段、全局筛滤器、图内筛选器、图表配置等功能;
  \item 权限管理: 对目录、工作表、仪表盘、图表、字段的读写权限管理;
  \item 工作表: 对内部数据和外部导入数据的简单可视化处理、SQL编辑创建合表;
  \item 移动端: 移动端仪表盘和图表的展示、图表的全局过滤和简单交互功能;
  \item 仪表盘: 包含仪表盘和图表的权限控制、添加图表、图表的拖拽布局等;
\end{itemize}
\end{onehalfspacing}

\datedsubsection{\textbf{实时计算}}{人人车, 2018.8-至今}
\begin{onehalfspacing}
实时计算平台(RTP)是一套基于\textit{Apache Spark Streaming}构建的\textbf{高性能实时大数据处理平台},主要针对\textbf{流式数据处理}场景。
\textbf{技术栈:}Vue, VueRouter, Vuex, D3.js, ElementUI, Sass, Eslint, StyleLint, Codemirror
\begin{itemize}
  \item 开发: 基于\textit{codemirror}实现的\textit{SQL}编辑作业的高亮、撤销、恢复、查找、代码检查等功能,支持调试、调参、版本管理、回滚、对比等功能,支持\textit{UDF}、\textit{UDAF}、\textit{UDTF}的创建编辑和配置使用等功能;
  \item 运维: 在线运维,监控作业运行状态的运行曲线,查看作业参数,展示作业SQL的血缘关系图;
\end{itemize}
\end{onehalfspacing}

% \datedsubsection{\textbf{即席查询}}{人人车, 2018.9-至今}
% \begin{onehalfspacing}
% 即席查询平台是一个自助取数平台,支持用户根据自己的需求,编写对应的\textit{sql}查询语句,交由系统执行最终生成查询结果。
% 满足用户随机性,多发性的查询需求;对用户查询进行权限控制;对用户维度和数据表维度查询的详细信息监控。
% \end{onehalfspacing}

% \datedsubsection{\textbf{数据采集}}{人人车, 2018.11-至今}
% \begin{onehalfspacing}
% 数据采集平台是对数据采集到输出和同步的过程规范化,简洁化,流程化的管理平台。
% \end{onehalfspacing}

% \datedsubsection{\textbf{ETL}}{人人车, 2019.1-至今}
% \begin{onehalfspacing}
% ETL是人人车大数据系统的完整数流程中的重要一环,ETL从多样的异构数据源中抽取到原始数据层后进行清洗、转换、集成,最后加载到数据仓库中,成为数据分析、数据挖掘的基础。
% ETL平台具备离线数据计算能力,以及周期调度能力,以实现不同数据源的跨平台整合。
% \end{onehalfspacing}

\datedsubsection{\textbf{Anne's Tibers}}{人人车, 2019.11-至今}
\begin{onehalfspacing}
人人车 VUE 中台最佳实践,效率提升百分百。基于 ElementUI 继续向上构建,提炼出典型模板/业务组件和区块等的物料系统。我们的技术栈 基于 Vue, ElementUI, anne, Scss, Eslint, StyleLint。
\textbf{技术栈:}Vue, ElementUI, Sass, Eslint, StyleLint
\begin{itemize}
  \item 负责表格相关的9个区块、拖拽相关的3个区块、tab模式筛选2个区块等区块功能开发;
  \item 负责部分组件的基于ElementUI继续向上构建和维护;
  \item 负责物料系统的部分稳定编写;
\end{itemize}
\end{onehalfspacing}


\datedsubsection{\textbf{BI商务智能系统}}{上海景栗科技有限公司, 2017.4-2018.1}
\begin{onehalfspacing}
实现了面向特定用户的\textbf{可定制敏捷数据可视化分析平台},支持业务人员进行自助式探索分析,支持参考线、下钻、联动、表计算等强大分析功能,为\textbf{企业级大数据分析}量身打造
\textbf{技术栈:} Backbone, Echarts.js, D3.js, grunt
\begin{itemize}
  \item 在\textit{saiku}开源代码上的前端重构, 增加仪表盘、数据卡片、数据源等模块;
  \item 丰富的仪表盘功能: 静态、动态文本,全局过滤器,图片,多种类型图表等;
  \item 可视化图表: 基于\textit{D3.js}和\textit{Echarts.js}的多种可视化图表;
  \item 前端性能优化: 首页加载,数据缓存等的优化;
  \item 基于\textit{openRefine}的后台,实现对数据的导入,清理过滤的功能
\end{itemize}
\end{onehalfspacing}

\datedsubsection{\textbf{SASS系统}}{上海景栗科技有限公司, 2016.10--2017.3}
\begin{onehalfspacing}
开发了基于\textit{\textbf{react}}的SASS系统的部分模块, 实现了\textbf{微信群运营和监控系统}的前端开发,该系统用于中小型企业微信社群的运营。
\textbf{技术栈:} React, Webpack, D3.js, JQuery
\begin{itemize}
  \item 基于\textit{D3}和\textit{g-cloud}词云算法的文本可视化;
  \item 基于\textit{JQuery}和\textit{React}的群管理、微任务管理和朋友圈投放管理模块的前端开发
\end{itemize}
\end{onehalfspacing}

% \datedsubsection{\textbf{软件开发工程师}}{上海海隆软件有限公司, 2015.9-2016.9}
% \begin{onehalfspacing}
% 日本花旗银行系统的维护和更新
% \begin{itemize}
%   \item 概要设计书、详细设计书、单元测试报告等的编写和维护
%   \item 子系统项目的前端开发
% \end{itemize}
% \end{onehalfspacing}

\section{\faUsers\ 开源项目}

\datedsubsection{\textbf{瓦力(\linkedin [官网]{http://walle-web.io})}}{前端, 2018.9-至今}
\begin{onehalfspacing}
\textit{Walle}是一个高体验,高颜值的\textbf{开源上线部署系统},支持自由配置项目,支持git、多用户、多语言、多项目、多环境同时部署的。\textit{Walle}让用户的代码发布终于可以不只能选择\textit{jenkins}。支持各种web、php、java、python、go等代码的发布、回滚可以通过web来一键完成。
\textbf{技术栈:} Vue, VueRouter, Vue CLI, ElementUI, G2, socket.io
\end{onehalfspacing}

\section{\faCogs\ IT 技能}
\begin{itemize}[parsep=0.5ex]
  \item 熟练使用\textit{react, vue}等框架;
  \item 熟练使用\textit{D3.js}和\textit{ECharts.js}和\textit{zrender};
  \item 熟悉\textit{git}协同工作;
  \item 熟悉\textit{webpack、grunt}等构建工具;
  \item 了解\textit{mocha、should、karma}的前端单元测试技术;
  \item 了解\textit{travis-ci、coveralls.io}做CI测试;
  \item 了解\textit{mondrian}的\textit{schema}和MDX
\end{itemize}

% \section{\faInfo\  2018总结}
% \begin{itemize}
%   \item 参与Walle开源项目的前端开发与运营;
%   \item 在公司内部做过3次分享(D3入门、swagger推广、vue数据双向绑定源码分析);
%   \item 发布8篇技术博客;
% \end{itemize}

% \section{\faInfo\  自我评价}
% \begin{itemize}
%   \item 积极主动,有极强的执行力、自我驱动力;
%   \item 做事稳重有条理,有担当,有能力;
% \end{itemize}

\end{document}
